\documentclass{beamer}
\usepackage[utf8]{inputenc}
\usepackage[brazil]{babel}
\usepackage{beamerthemesplit}
\usepackage{url}
\usepackage{longtable}
\usepackage{textcomp}
\usepackage{boxedminipage}
%% Pacotes Usados para exemplos
\usepackage{fancyvrb}
\usepackage{listings}
\usepackage{enumerate}


\newcommand{\pacote}[1]{\texttt{\textcolor[rgb]{0.3,0.6,0}{#1}}}
% \newenvironment{semiverbatim}{\begin{semiverbatim}\color{blue}}{\color{black}\end{semiverbatim}}
% \renewenvironment{verbatim}{\color{blue}\begin{verbatim}}{\end{verbatim}\color{black}}
% \newenvironment{block}{\vspace{8pt}\begin{minipage}{\textwidth}}{\end{minipage}\vspace{8pt}}
\newenvironment{fblock}{\vspace{8pt}\begin{boxedminipage}{\textwidth}}{\end{boxedminipage}\vspace{8pt}}


\title{Continuando aula do \LaTeXe}
\author{Leonardo Cisneiros}
\institute{Universidade Federal Rural de Pernambuco \\ Unidade Acadêmica de Serra Talhada}
\date{\today}
% \setbeamercovered{dynamic}
\begin{document}


\frame{\titlepage}

\section[Outline]{}
\frame{\tableofcontents}

\input{corposflutuantes}
%%%%%%%%%%%%%%%%%%%%%%%%%
%%%%%%%%%%%%%%%%%%%%%%%%%

\section{Índice Remissivo}

\begin{frame}[fragile]
\frametitle{O programa \texttt{makeindex}}

Para usar o makeindex, inclua no preâmbulo do seu documento os seguintes comandos:

	\begin{semiverbatim}
		\\usepackage\{makeidx\}
		\\makeindex
	\end{semiverbatim}

\end{frame}

\begin{frame}[fragile]
\frametitle{Marcando as entradas do índice}

	Para indexar uma palavra use o comando \verb!\index{...}! com a palavra como parâmetro. Ex:

	\begin{quotation}

	 {\itshape No século XVI, monges\verb!\index{monge}! 
	de uma cidade belga, motivados por uma 
	epidemia\verb!\index{epidemia}! de cólera\verb!\index{cólera}!,
	 proibiram o consumo de água\verb!\index{H_{2}O}! e 
	obrigaram a população a tomar somente cerveja\verb!\index{cerveja}!.}

	\end{quotation}

\end{frame}

\begin{frame}[fragile]
\frametitle{Firulas}

	O comando \verb!\index! permite firulas mais interessantes:

\begin{small}
\begin{tabular}{l | l}
\hline
\textbf{Código} 		& \textbf{Resultado} \\
\hline
\verb$\index{cerveja!produção}$  & cerveja, 32, 45, 118-120 \\
							& \hspace{15pt} produção, 46 \\
\hline
\verb$\index{cerveja|(}$		&	\\
\verb$\index{cerveja|)}$		& cerveja, 8--15  \\
\hline
\verb$\index{cerveja|ver {fermentados}}$ & cerveja, \emph{ver} fermentados \\ \\
\hline
\verb$\index{cerveja@\textbf{cerveja}}$ & \textbf{cerveja, 32} \\
\hline 
\verb$\index{cerveja|textbf}$ & cerveja, \textbf{32} \\
\hline
\end{tabular}
\end{small}

\end{frame}

\begin{frame}[fragile]
\frametitle{Criando o índice}

\begin{itemize}
 \item Primeiro processa-se o arquivo .tex usando o programa \pacote{makeindex}: \\
	\begin{semiverbatim}
	 makeindex \emph{nomedoarquivosemextensão}
	\end{semiverbatim}
 \item Coloca-se o comando \verb!\printindex! no ponto do documento onde o índice deve aparecer.
\item Processa-se o documento com o \LaTeX novamente. 
	
\end{itemize}


\end{frame}

\begin{frame}[fragile]
\frametitle{Funcionalidades adicionais}

	\begin{itemize}
	 \item O programa e pacote \pacote{authorindex} cria índices dos autores citados no documento indicando as páginas onde eles foram citados.
	\item O pacote \pacote{index} estende as funcionalidades para índices, permitindo, por exemplo, múltiplos índices 
	\item O programa \pacote{addindex} cria automaticamente as entradas de índices no texto a partir de uma lista de nomes.
	\item O programa \pacote{forindex} faz algo similar com uma lista de formatação mais simples (mas mais flexível). É escrito numa linguagem esquisita (snobol4).
	\item Assim ainda dá algum trabalho, por isso o faltava era um script para extrair essa lista de palavras relevantes de uma amostra de textos. Não é difícil de fazer. Fica como exercício.. ;-)
	\end{itemize}


\end{frame}

\section{Apimentando as coisas}


\subsection{Cabeçalhos e Rodapés}

\begin{frame}[fragile]
\frametitle{O comando nativo}

O \LaTeX tem um comando nativo para modificar cabeçalhos: \verb!\pagestyle{...}!

As opções são: 

\begin{itemize}
 \item \texttt{plain} - só o número da página
\item \texttt{empty} - Cabeçalho e rodapé vazios, sem número de páginas
\item \texttt{headings} - Coloca cabeçalhos em cada página definidos pelo estilo do documento.
\item \texttt{myheadings} - Você especifica o cabeçalho com a ajuda dos comandos \verb!\markboth{...}{...}! ou \verb!\markright{..}!
\end{itemize}
\vspace{15pt}
O Comando \verb!\thispagestyle{...}! permite definir o estilo só para a página atual.

\end{frame}

\begin{frame}[fragile]
\frametitle{O pacote \texttt{fancyhdr}}

O pacote \pacote{fancyhdr} dá mais poderes de formatação sobre cabeçalhos e rodapés, podendo redefinir cada um dos seguintes elementos da página:

\begin{center}
% use packages: array
\begin{fblock}
\noindent\makebox[\textwidth]{Cabeçalho Esq.\hfill
Cabeçalho Centro\hfill Cabeçalho Dir.}
\noindent\makebox[\textwidth]{\hrulefill}\\[\baselineskip]
\noindent\makebox[\textwidth]{\hfill corpo do texto \hfill}\\[\baselineskip]
\noindent\makebox[\textwidth]{\hrulefill}
\noindent\makebox[\textwidth]{Rodapé Esq.\hfill
Rodapé Centro\hfill Rodapé Dir.}
\end{fblock}

\end{center}
\end{frame}

\begin{frame}[fragile]
\frametitle{Um exemplo}
\begin{small}
\begin{flushleft}
\begin{fblock}
\noindent\makebox[\textwidth]{\alert<2>{Seção 2} \hfill
\hfill \alert<3>{\textbf{A tensão pré-VA}}}
\noindent\makebox[\textwidth]{\hrulefill}\\[\baselineskip]
\noindent\makebox[\textwidth]{\hfill corpo do texto \hfill}\\[\baselineskip]
\noindent\makebox[\textwidth]{\hrulefill}
\noindent\makebox[\textwidth]{\alert<4>{\itshape \tiny \textcopyleft Leonardo Cisneiros}\hfill
\alert<5>{\scshape \tiny Serra Talhada, MMVIII} \hfill \alert<6>{23}}
\end{fblock}
\end{flushleft}
\end{small}

\begin{footnotesize}
\begin{semiverbatim}
\\pagestyle\{fancy\}
\alert<2>{\\lhead\{Seção \\thesection\}}
\\chead\{\}
\alert<3>{\\rhead\{\textbf\{A Tensão pré-VA\}\}}
\alert<4>{\\lfoot\{\{\\itshape \\tiny \\textcopyleft Leonardo Cisneiros\}\}}
\alert<5>{\\cfoot\{\\scshape \\tiny Serra Talhada, 2008\}}
\alert<6>{\\rfoot\{\\thepage\}}
\end{semiverbatim}
\end{footnotesize}

\end{frame}

\begin{frame}[fragile]
\frametitle{Linhas no cabeçalho e rodapé}

Para definir linhas separando o cabeçalho e o rodapé no corpo de texto, use:

\begin{itemize}
 \item Para o cabeçalho: \\ 
	\verb!\renewcommand{\headrulewidth}{0.4pt}!
\item Para o rodapé: \\
	\verb!\renewcommand{\footrulewidth}{0.4pt}!
\end{itemize}

O segundo argumento define a espessura da linha. 

\end{frame}

\begin{frame}[fragile]
\frametitle{Mais controle}

Algumas classes têm paginação dos dois lados, isto é, as páginas pares e ímpares têm layouts diferentes. Para definir cabeçalhos e rodapés diferenciados use os seguintes comandos:

\begin{itemize}
 \item \verb!\fancyhead[RO, LE]{texto}! para cabeçalhos
\item \verb!\fancyfoot[LE,RO]{texto}! para rodapés
\item \verb!\fancyhf[LEH,ROF]{texto}! para os dois
\end{itemize}

As letras maiúsculas se referem à posição definida:

\begin{center}
\begin{footnotesize}
 \begin{tabular}{|c|c|}
 \hline
E & Página Par \\
O & Página Ímpar \\ 
\hline
L & Campo Esquerdo \\
C & Campo Central \\
R & Campo Direito \\ 
\hline
H & Cabeçalho \\
F & Rodapé \\ \hline
\end{tabular}
\end{footnotesize}
\end{center}

\end{frame}

% \begin{frame}[fragile]
% \frametitle{O exemplo anterior}
% 
% \begin{small}
% \begin{flushleft}
% \begin{fblock}
% \noindent\makebox[\textwidth]{\alert<2>{Seção 2} \hfill
% \hfill \alert<3>{\textbf{A tensão pré-VA}}}
% \noindent\makebox[\textwidth]{\hrulefill}\\[\baselineskip]
% \noindent\makebox[\textwidth]{\hfill corpo do texto \hfill}\\[\baselineskip]
% \noindent\makebox[\textwidth]{\hrulefill}
% \noindent\makebox[\textwidth]{\alert<4>{\itshape \tiny \textcopyleft Leonardo Cisneiros}\hfill
% \alert<5>{\scshape \tiny Serra Talhada, MMVIII} \hfill \alert<6>{23}}
% \end{fblock}
% \end{flushleft}
% \end{small}
% 
% \begin{semiverbatim}
% \\fancyhead[ 
% \\fancyfoot[L
% 
% \end{semiverbatim}
% 
% 
% 
% \end{frame}


%%%%%%%%%%%%%%%%%%%%%%%
%%%%%%%%%%%%%%%%%%%%%%%

\subsection{Definindo seus comandos}

\subsection{Fontes}

\subsection{Alterando Numeração}

\begin{frame}[fragile]
\frametitle{Contadores}

Toda numeração automática do \LaTeX tem um contador associado:
\begin{center}
% use packages: array

\begin{tabular}{llll}

 part & paragraph & figure & enumi \\ 
chapter & subparagraph & table & enumii \\ 
section & page & footnote & enumiii \\ 
subsection & equation & mpfootnote & enumiv \\ 
subsubsection &  &  & 
\end{tabular}
\end{center}
\end{frame}

\begin{frame}[fragile]
 \frametitle{Imprimindo o contador}

Evoque o contador como argumento de algum desses comandos de formatação:
\begin{tabular}{lll}
\verb!\roman{contador}! & \verb!\alph{contador}! & \verb!\arabic{contador}!  \\
\verb!\Roman{contador}! & \verb!\Alph{contador}! &
\end{tabular}
Ex.: para imprimir o número da página em algarismos romanos minúsculos (sabe-se lá para quê), use
\verb!\roman{page}!

\end{frame}

\begin{frame}[fragile]
 \frametitle{Redefinindo o contador}
Um uso mais interessante é redefinir o formato do contador. 

O comando que o \LaTeX realmente usa para imprimir o contador tem a forma \verb!\the...! seguido
do nome do contador, como em \verb!\thefootnote!. Redefina-o como qualquer comando, usando o comando \verb!\renewcommand!:

\begin{semiverbatim}
\renewcommand{\thesection}{\Roman{section}} 
\end{semiverbatim}

Redefine a numeração de seção para números romanos maiúsculos.

\end{frame}

\begin{frame}[fragile]
 \frametitle{Um exemplo mais complicado}

Redefinindo mais fundo a hierarquia de seções:

\begin{small}
\verb!\renewcommand{\thesection}{\Roman{section}}!
\verb!\renewcommand{\thesubsection}!\ldots\\
\hspace{100pt}\ldots \verb!{\thesection.\roman{subsection}}!
\verb!\renewcommand{\thesubsubsection}!\ldots\\
\hspace{30pt}\ldots\verb!{\thesection.\thesubsection:\alph{subsubsection}}!
\end{small}
\linebreak
As subseções ficariam indicadas por I.i, I.ii, II.iv e as subsubseções por I.ii:a ou V.ix:k

\end{frame}


\section{Pacotes e mais pacotes}

\begin{frame}[fragile]
\frametitle{Pacote \texttt{url}}

Pacote para imprimir URLs, emails e outras URIs. É como o \verb!\verb!, só que permite quebra de linha para as URLs longas. 

\begin{semiverbatim}
\\usepackage\{url\}
\vdots 
\\url\{http://groups.google.com/group/metodologia-si/\}
\end{semiverbatim}
\url{http://groups.google.com/group/metodologia-si/} 
\end{frame}

\begin{frame}[fragile]
\frametitle{Customizando listas}

O pacote \pacote{enumerate} permite configurar a numeração de listas de maneira simples:


	\begin{columns}
		\column[t]{.5\textwidth}
			\begin{footnotesize}
			\begin{semiverbatim}
			\\begin\{enumerate\}[EX 1:]
			\\item Primeiro exemplo
			\\item Segundo exemplo
				\\begin\{enumerate\}[1$^{a}$ Obs.]
				\\item lalala
				\\item lalala
				\\end\{enumerate\}
			\\item Terceiro exemplo
			\\end\{enumerate\}
			\end{semiverbatim}
			\end{footnotesize}

		\column[t]{.5\textwidth}
			\begin{small}
			\begin{enumerate}[EX 1:]
			\item Primeiro exemplo
			\item Segundo exemplo
				\begin{enumerate}[1$^{a}$ Obs.]
				\item lalala
				\item lalala
				\end{enumerate}
			\item Terceiro exemplo
			\end{enumerate}
			\end{small}
	\end{columns}



\end{frame}


\end{document}
    