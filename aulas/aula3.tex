\documentclass[a4paper,11pt]{scrartcl}
\usepackage[T1]{fontenc}
\usepackage[utf8]{inputenc}
\usepackage[brazil]{babel}
\usepackage{amsfonts}
\usepackage{amsmath}
\usepackage{amsfonts}
\usepackage{amssymb}
\usepackage{amsthm}
\usepackage{xypic}
\usepackage{enumitem}
\usepackage[mathscr]{euscript}
\usepackage{bussproofs}
\usepackage{color}
\usepackage{graphics}
\usepackage{microtype}
\usepackage{times}
% \usepackage{libertine}

\renewcommand{\ttdefault}{cmtt}

\usepackage{supertabular}
\usepackage{indentfirst}

\usepackage{url}
\usepackage{multirow}
\renewcommand{\sfdefault}{ugq}


%%%%%%%%%%%%%%%%%%%%%%%%%%%%%%%%%%%%%%%%%%%
% Pacote e definições de manual para o título de seção afrescalhado
\usepackage{titlesec}
\newcommand\secformat[1]{%
\parbox[b]{.5\textwidth}{\filleft\bfseries #1}%
\quad\rule[-12pt]{2pt}{50pt}\quad
{\fontsize{45}{45}\selectfont\thesection}}
\titleformat{\section}[block]%
{\filleft\normalfont\sffamily}{}{0pt}{\secformat}
\titlespacing*{\section}{0pt}{*3}{*2}[1pc]
%%%%%%%%%%%%%%%%%%%%%%%%%%%%%%%%%%%%%%%%%%%



%opening
\title{Novos Elementos de \LaTeX}
\author{Leonardo Cisneiros}
\newcommand{\dummytext}{%
Die menschliche Vernunft has das besondere Schicksal in einer
Gattung ihrer \textsc{Erkenntnisse}: \textit{da\ss{} sie durch
Fragen belästigt wird, die sie nicht abweisen kann}, denn sie sind
ihr durch die Natur der Vernunft selbst aufgegeben, \textbf{die sie
aber auch nicht beantworten kann}, denn sie übersteigen alles
Vermögen der menschlichen Vernunft
}


%%%%%%%%%%%%%%%%%%%%%%%%%%%%%%%%%%%%%%%
% Macro para gerar as amostras das fontes
\newcommand{\fontsample}[2]{%
{\fontfamily{#1}\selectfont
	\begin{tabular*}{\textwidth}{p{0.65\textwidth}p{
0.35\textwidth}}
		{\large\sffamily #2 (#1)}	& 
							\\\hline
			&				 \\

		ABCDEFGHIJKLMNOPQRSTUVWXYZ  & 
		\multirow{2}{7mm}{\fontsize{35}{40}\selectfont
AaBbCc} \\
		abcdefghijklmnopqrstuvwxyz1234567890 &  \\
		\textbf{The Quick Brown Fox Jumps Over the Lazy Dog}
&
		\multirow{2}{7mm}{\fontsize{35}{40}\selectfont
		FfGgKk} \\
		\textit{The Quick Brown Fox Jumped Over the Lazy Dog}
& \\
			&  \\
		\parbox{0.65\textwidth}{\footnotesize\dummytext} &
		\parbox{0.35\textwidth}{\fontsize{35}{40}\selectfont
		MmRrQq\\ SsTtXx} \\
	\end{tabular*}\vspace{3em}\par
	}
}

\begin{document}

\maketitle


\section{Controlando a Fonte}

\subsection{Comandos de Baixo Nível}

Toda fonte no \LaTeX tem cinco atributos:

\begin{description}
\setlength{\itemsep}{0pt} 
\item[Codificação] Especifica a ordem que os caracteres aparecem na
fonte. Você só precisa realmente mexer com isso quando se tratar de
caracteres mais esquisitos e outros alfabetos como grego, cirílico
etc.
\item[Família] O nome de uma coleção de fontes. Um design de fonte.
Quando falamos em \emph{uma} fonte na verdade estamos falando de uma
\emph{família} de fontes, pois na tipografia correta, as fontes para
o itálico, o negrito, o small caps, por exemplo, são todas
distintas, ainda que compartilhem diversos elementos comum de design.
\item[Série] Indica o peso ou a expansão da fonte. O negrito, p.ex.,
é uma série.
\item[Forma] Indica a forma da fonte dentro da família. O itálico é
uma forma.
\item[Tamanho] Auto-explicativo\ldots
\end{description}

Os comandos de mais baixo nível no \LaTeX para definir esses
atributos são os seguintes:

\begin{description}
\setlength{\itemsep}{0pt}
 \item[Codificação da fonte] \verb!\fontencoding{xx}!
 \item[Família da Fonte] \verb!\fontfamily{xxx}!
 \item[Série da Fonte] \verb!\fontseries{xx}!
 \item[Formato da Fonte] \verb!\fontshape{xx}!
 \item[Tamanho da Fonte] \verb!\fontsize{xx}{xx}!
\end{description}

Depois de todos os comandos de definição da fonte, use o comando
\verb!\selectfont! para começar a usá-la. Também lembre-se que o
comando \verb!\selectfont! faz com que a seleção valha até ser
revertida ou até encontrar um ``\}'', indicando o final de um grupo. 

As séries são: \texttt{m} Médio, \texttt{b} Negrito,
\texttt{bx}
Negrito estendido, \texttt{sb} Semi-negrito, \texttt{c}
condensado.\marginpar{\footnotesize Série}
Essas distinções só funcionam se a fonte tiver todos esses formatos
pre-definidos, o que é difícil nas fontes disponíveis nas
distribuições normais do \LaTeX e talvez só seja encontrado em
fontes profissionais bem caras. Assim, normalmente só usaremos o
negrito normal e aí é melhor usar os comandos \verb!\bfseries! e
\verb!\textbf{}! para negrito e os comandos \verb!\mdseries! e
\verb!\textmd{}! para voltar ao texto normal.

Os valores comuns para o formato da fonte são
\texttt{n} Normal,
\texttt{it} \textit{Itálico}, \texttt{sl}
\textsl{Inclinado}\footnote{Note que o itálico não é só a
inclinação do texto, pois o desenho da fonte é diferente, como fica
claro na diferença da letra ``a''} e \texttt{sc}
\textsc{Versalete}.\marginpar{\footnotesize Forma}
Em todo caso para definir a forma da fonte no meio de um texto é
mais prático usar os comandos do \LaTeX: para itálico,
\verb!\textit{}! ou \verb!\itshape!; para inclinado,
\verb!\textsl{}! ou \verb!\slshape!; para versalete,
\verb!\textsc{}! ou \verb!\scshape!; e para texto normal (em pé),
\verb!\textup{}! ou \verb!\upshape!. 

Para tamanho também temos uma série de comandos
pré-definidos:
\verb!\tiny! 5pt ({\tiny amostra}), \verb!\scriptsize! 7pt
({\scriptsize amostra}), \verb!\footnotesize! 8pt ({\footnotesize
amostra}), \verb!\small! 9pt ({\small amostra}), \verb!\normalsize!
10pt ({\normalsize amostra}) ------\marginpar{\footnotesize{}Tamanho}
Para outros tamanhos de fonte,
você deve usar o comando de baixo nível, indicando no primeiro
parâmetro o tamanho da fonte em pt e no segundo, o tamanho da fonte
contando com um espaço entre linhas, o que pode ser importante em
alguns casos. Com esse comando você tem liberdade para definir
fontes menores que \verb!\tiny!:

\verb!\fontsize{3}{3}\selectfont!:
{\fontsize{3}{3}\selectfont Esse aqui é um texto realmente pequeno
que eu imagino que seja ilegível A NÃO SER QUE APELEMOS PARA
MAIÚSCULAS!}

E também pode definir letras bem maiores do que \verb!\Huge!.

Por exemplo, com \verb!\fontsize{70}{70}\selectfont!:

{\fontsize{70}{70}\selectfont Texto grande!}
\vspace{2em}

Codificação da fonte não é algo com o qual
você terá que lidar com
muita freqüência a não ser que se aventure em fazer um texto bilíngüe
em russo e aramaico.\marginpar{\footnotesize Codificação} No entanto,
convém ficar avisado da importância
da codificação quando a fonte se comportar de maneira inesperada e
alguns caracteres não forem renderizados ou o forem de maneira
esquisita.

O atributo mais importante e mais utilizado de maneira direta é o da
família da fonte.\marginpar{\footnotesize Família} As fontes padrão
do \TeX são nomeadas a partir de um esquema que permite embutir uma
certa classificação no nome do arquivo e mantê-lo no limite de 8
caracteres, necessário para garantir sua consistência nos diversos
sistemas de arquivos. Esses arquivos ficam na pasta
\texttt{./fonts/afm/} de sua distribuição \TeX/\LaTeX divididos
dentro dela pelas ``forjas'' que é como se chama quem fabrica uma
fonte, numa homenagem à época que as fontes eram feitas de chumbo
fundido. O primeiro caracter do nome da fonte indica a forja, os
outros dois indicam a família, os outros dois indicam a série e a
forma, os dois seguintes indicam a codificação e, por fim, um ``c''
no final indica se a fonte é small caps ou não. Por exemplo, o nome
da fonte \texttt{phvb8a}, encontrável em
\texttt{\$TEXMF/fonts/afm/adobe/helvetic/},indica que ela é uma fonte
da Adobe (\texttt{p}), Helvetica (\texttt{hv}), negrito (\texttt{b})
de codificação ANSI 8-bits (\texttt{8a}). Porém, para a especificação
da família bastam os três primeiros caracteres. Assim, se você quiser
{\fontfamily{phv}\selectfont usar a fonte Helvetica, no meio do
texto, como agora, basta usar o comando
\verb!\fontfamily{phv}\selectfont! no início do grupo e
\emph{alterações na forma} \textbf{ou na série}, como essas, serão
feitas a partir da família.}



\subsection{Definindo fontes padrão}

O \LaTeX tem uma série de comandos que carregam as definições padrão
para as fontes e que podem ser modificados com o comando
\verb!\renewcommand{}{}!. O comando \verb!\rmdefault! guarda a
definição da fonte serifada que é usada como a padrão para todo
documento. Da mesma forma, o comando \verb!\sfdefault! guarda a
definição da fonte sem-serifa e o \verb!\ttdefault!, o da fonte de
tamanho fixo. Para redefinir a fonte serifada para Adobe Times, por
exemplo, use \verb!\renewcommand{\rmdefault}{ptm}! e para definir a
fonte sem serifa para URW Grotesque, como é o caso nesse documento,
use \verb!\renewcommand{\sfdefault}{ugq}!.

Outros comandos/variáveis que definem padrões relativos a fontes
são:\vspace{1em}

\begin{tabular}[l]{rl}
 \verb!\encodingdefault! & A codificação padrão da fonte principal do
texto \\
 \verb!\familydefault! & A família padrão da fonte principal do texto
\\
 \verb!\seriesdefault! & A série padrão da fonte principal do texto
\\
 \verb!\shapedefault! & A forma padrão da fonte principal do texto \\
 \verb!\itdefault! & A fonte padrão para o comando \verb!\textit! \\
 \verb!\bfdefault! & A fonte padrão para o comando \verb!\textbf! \\
 \verb!\mddefault! & A fonte padrão para o comando \verb!\textmd! \\
 \verb!\sldefault! & A fonte padrão para o comando \verb!\textsl! \\
 \verb!\scdefault! & A fonte padrão para o comando \verb!\textsc! \\
\end{tabular}

\subsection{Pacotes}

A definição do conjunto de fontes padrão de um documento pode ser
feita de maneira muito mais simples através da invocação de pacotes.
O pacote \texttt{times} muda o trio \texttt{rm-sf-tt} para Times,
Helvetica e Courier, respectivamente. O pacote \texttt{newcent},
muda para New Century Schoolbook, Helvetica e Courier. O pacote
\texttt{palatino}, para Palatino, Avant Garde e Courier. Para
ficarmos só nas fontes da Adobe, temos ainda os seguintes pacotes:
\texttt{bookman}, \texttt{avant}, \texttt{courier} e
\texttt{zapfchan}. A melhor maneira de descobrir outros é navegando
pelo \textsc{The \LaTeX{} Font Catalogue}:
\url{http://www.tug.dk/FontCatalogue/}. 

\section{Amostras de Fonte}

\noindent\fontsample{ccr}{Computer Modern Roman}

\noindent\fontsample{ptm}{Adobe Times New Roman}

\noindent\fontsample{ppl}{Adobe Palatino}

\noindent\fontsample{pbk}{Adobe Bookman Old Style}

\noindent\fontsample{phv}{Adobe Helvetica}

\noindent\fontsample{pag}{Adobe Avant Garde}

\noindent\fontsample{pnc}{Adobe New Century Schoolbook}

\noindent\fontsample{pzc}{Adobe Zapf Chancery}

\noindent\fontsample{bch}{Bitstream Charter}

\noindent\fontsample{ugq}{URW Grotesque}

\noindent\fontsample{pag}{Adobe Avant Garde}

Mais amostras de fontes e informações de como instalá-las e usá-las
podem ser encontradas no \textsc{The \LaTeX{} Font Catalogue}:
\url{http://www.tug.dk/FontCatalogue/}. Informações detalhadas sobre
como criar os arquivos de configuração para instalar fontes que não
estejam no CTAN podem ser encontradas no documento
\texttt{fontinstallationguide}. É aquela coisa: instalar fontes no
\TeX é bom, mas morrer queimado é muito melhor. No entanto, para os
interessados (e possuidores de arquivos de fontes profissionais),
pode valer muito a pena e significar um resultado perfeito e
profissional. A via mais simples para aumentar a base de fontes do
\TeX de maneira a permitir o uso de fontes do sistema sem nenhuma
instalação especial é através do XeTeX/XeLaTeX. Mas atentem que usar
qualquer fonte vagabunda encontrada na internet pode acabar com a
qualidade tipográfica do documento. A graça de usar o XeLaTeX está
em poder usar \emph{fontes profissionais} de maneira simples.

\input{definindodimensoes}

\end{document}
