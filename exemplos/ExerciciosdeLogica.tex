\documentclass[a4paper,10pt]{article}
\usepackage[utf8]{inputenc}

\usepackage{amsfonts}
\usepackage{amsmath}
\usepackage{amsfonts}
\usepackage{amssymb}
\usepackage{amsthm}
\usepackage{xypic}
\usepackage{enumitem}
\usepackage[mathscr]{euscript}
\usepackage{bussproofs}
\usepackage{color}
\usepackage{graphics}
\usepackage{multicol}

\usepackage[brazil]{babel}
\usepackage{microtype} % pacote para ajuste fino de espaçamento, para melhor alinhamento
%\usepackage[urw-garamond]{mathdesign} % fonte Garamond com fonte para matemática, tem que instalar do CTAN
 \usepackage{mathpazo} % fonte Palatino com suporte para matemática

\usepackage{sectsty} % Formatação da fonte das seções
\allsectionsfont{\fontfamily{phv}\fontseries{b}\selectfont}

\newtheorem{exrcc}{Exercício}[subsection] %Definição de um ambiente para os enunciados de exercício
\newtheorem{exmpl}{Exemplo}[subsection] %Definição de um ambiente para exemplos numerados



\title{\fontfamily{phv}\fontsize{28}{28}\fontseries{b}\selectfont%Helvetica Bold tamanho 28 para o título
Exercícios de Lógica Proposicional}
\author{Prof. Leonardo Cisneiros}

\begin{document}

\maketitle

\section{Folha de Cola}

% A formatação dessa seção está péssima, precisa de ajustes no alinhamento.


\subsection{Tabelas-verdade}

\begin{multicols}{2}

\textbf{Negação}\\
\begin{tabular}{c|c}
 $\phi$ & $\neg\phi$ \\
 \hline
 V	&	F \\
 F	&	V \\
\end{tabular}


\textbf{Implicação}\\
\begin{tabular}{c|c|c}
$\phi$ & $\rightarrow$ & $\psi$ \\
 \hline
 V	&	V	&	V  \\
 V	&	F	&	F \\
 F	&	V	&	V \\
 F	&	V	&	F
\end{tabular}



\textbf{Disjunção inclusiva}\\
\begin{tabular}{c|c|c}
$\phi$ & $\vee$ & $\psi$ \\
 \hline
 V	&	V	&	V  \\
 V	&	V	&	F \\
 F	&	V	&	V \\
 F	&	F	&	F
\end{tabular}

\textbf{Conjunção}\\
\begin{tabular}{c|c|c}
$\phi$ & $\wedge$ & $\psi$ \\
 \hline
 V	&	V	&	V  \\
 V	&	F	&	F \\
 F	&	F	&	V \\
 F	&	F	&	F
\end{tabular}

\textbf{Equivalência}\\
\begin{tabular}{c|c|c}
$\phi$ & $\leftrightarrow$ & $\psi$ \\
 \hline
 V	&	V	&	V  \\
 V	&	F	&	F \\
 F	&	F	&	V \\
 F	&	V	&	F
\end{tabular}

\end{multicols}

\subsection{Regras de Inferência}

\subsubsection{Regras Básicas}
As regras mais básicas relativas a um operador são as que dizem o que garante a
afirmação de uma frase com aquele operador como operador principal (regras de
introdução) e as que dizem o que pode ser inferido de uma frase com aquele operador
como operador principal (regras de eliminação).

\begin{multicols}{2}

\textbf{Introdução da Implicação}


\alwaysNoLine
\AxiomC{[$\phi$]}
\UnaryInfC{\vdots}
\UnaryInfC{$\psi$}
\alwaysSingleLine
\RightLabel{\scriptsize $\rightarrow{}I$}
\UnaryInfC{$\phi\rightarrow\psi$}
\DisplayProof

\textbf{Modus Ponens \emph{ou} Eliminação da Implicação}


\AxiomC{$\phi\rightarrow\psi$}
\AxiomC{$\phi$}
\RightLabel{\scriptsize $\rightarrow{}E$}
\BinaryInfC{$\psi$}
\DisplayProof

\textbf{Introdução da Conjunção}


\AxiomC{$\phi$}
\AxiomC{$\psi$}
\RightLabel{\scriptsize $\wedge{}I$}
\BinaryInfC{$\phi\wedge\psi$}
\DisplayProof

\textbf{Eliminação da Conjunção}


\AxiomC{$\phi\wedge\psi$}
\RightLabel{\scriptsize $\wedge{}E$}
\UnaryInfC{$\phi$}
\DisplayProof

\textbf{Introdução da Disjunção}
\AxiomC{$\phi$}
\RightLabel{\scriptsize $\vee{}I$}
\UnaryInfC{$\phi\vee\psi$}
\DisplayProof

% \textbf{Eliminação da Disjunção}\footnote{A lógica é: se você pode derivar a mesma
% consequência das duas partes da disjunção então tanto faz qual das duas é
% verdadeira, você pode passar direto para a consequência.}
% \AxiomC{$\phi$}
% \

\textbf{Introdução da Negação}\footnote{O símbolo $\bot$ representa uma contradição, um
absurdo. A regra diz que se de uma hipótese você é capaz de derivar uma contradição,
então você pode concluir a negação dessa hipótese. Essa contradição pode ser uma
contradição absoluta, mas também uma contradição com uma premissa dada. Um exemplo de
uso dessa regra é a justificação da regra do \emph{modus tollens} --
$\{\phi\rightarrow\psi, \neg\psi\}\vdash\neg\phi$ -- a partir da introdução da negação e
da introdução da implicação:

\AxiomC{$\phi\rightarrow\psi$}
\AxiomC{[$\phi$]$^{1}$}
\BinaryInfC{$\psi$}
\AxiomC{$\neg\psi$}
\BinaryInfC{$\bot$}
\UnaryInfC{$\neg\phi$}
\DisplayProof

}

\alwaysNoLine
\AxiomC{[$\phi$]}
\UnaryInfC{\vdots}
\UnaryInfC{$\bot$}
\alwaysSingleLine
\RightLabel{\scriptsize $\neg{}I$}
\UnaryInfC{$\neg\phi$}
\DisplayProof

\textbf{Introdução da Equivalência}

\AxiomC{$\phi\rightarrow\psi$}
\AxiomC{$\psi\rightarrow\phi$}
\RightLabel{\scriptsize $\leftrightarrow{}I$}
\BinaryInfC{$\phi\leftrightarrow\psi$}
\DisplayProof

\textbf{Eliminação da Equivalência}\footnote{Você pode derivar qualquer uma das
implicações. Aqui só aparece uma delas, por causa do espaço}

\AxiomC{$\phi\leftrightarrow\psi$}
\RightLabel{\scriptsize $\leftrightarrow{}E$}
\UnaryInfC{$\phi\rightarrow\psi$}
\DisplayProof

\end{multicols}

\subsubsection{Regras Auxiliares}

\begin{multicols}{2}
{
\textbf{Modus Tollens}

\AxiomC{$\phi\rightarrow\psi$}
\AxiomC{$\neg\psi$}
\RightLabel{\scriptsize $MT$}
\BinaryInfC{$\neg\phi$}
\DisplayProof
}

{
\textbf{Silogismo Disjuntivo}

\AxiomC{$\phi\vee\psi$}
\AxiomC{$\neg\phi$}
\RightLabel{\scriptsize $SD$}
\BinaryInfC{$\psi$}
\DisplayProof
}

{
\textbf{Silogismo Hipotético}

\AxiomC{$\phi\rightarrow\psi$}
\AxiomC{$\psi\rightarrow\sigma$}
\RightLabel{\scriptsize $SH$}
\BinaryInfC{$\phi\rightarrow\sigma$}
\DisplayProof
}

{
\textbf{Substituição de Equivalentes}\footnote{Só um atalho para a inferência completa
que seria:

\AxiomC{$\phi\leftrightarrow\psi$}
\RightLabel{\scriptsize $\leftrightarrow{}E$}
\UnaryInfC{$\phi\rightarrow\psi$}
\AxiomC{$\phi$}
\RightLabel{\scriptsize $\rightarrow{}E$}
\BinaryInfC{$\psi$}
\DisplayProof
}

\AxiomC{$\phi$}
\AxiomC{$\phi\leftrightarrow\psi$}
\RightLabel{\scriptsize $Eq$}
\BinaryInfC{$\psi$}
\DisplayProof
}

\end{multicols}

\subsubsection{Equivalências}

As equivalências da lista seguinte podem ser usadas como axiomas a fim de simplificar
as provas. Para simplificar bastante mesmo, eu aceitarei a inferência imediata entre
fórmulas equivalentes quando for assinalada a regra utilizada. Ver exemplo
\ref{bigdemo} abaixo.

\begin{multicols}{2}
\begin{description}
\setlength{\itemsep}{0pt}
 \item[De Morgan (DM)] $\neg{\phi\wedge\psi}\leftrightarrow{\neg\phi\vee\neg\psi}$
 \item[De Morgan (DM)] $\neg(\phi\vee\psi)\leftrightarrow(\neg\phi\wedge\neg\psi)$
 \item[Comutação (Com)] $(\phi\vee\psi)\leftrightarrow(\psi\vee\phi)$
 \item[Comutação (Com)] $(\phi\wedge\psi)\leftrightarrow(\psi\wedge\phi)$
 \item[Associação (Assoc)]
$(\phi\vee(\psi\vee\sigma))\leftrightarrow((\phi\vee\psi)\vee\sigma)$
 \item[Associação (Assoc)]
$(\phi\vee(\psi\wedge\sigma))\leftrightarrow((\phi\wedge\psi)\vee\sigma)$
\item[Distribuição (Dist)]
$(\phi\wedge(\psi\vee\sigma))\leftrightarrow((\phi\wedge\psi)\vee(\phi\wedge\sigma))$
\item[Distribuição (Dist)]
$(\phi\vee(\psi\wedge\sigma))\leftrightarrow((\phi\vee\psi)\wedge(\phi\vee\sigma))$
 \item[Dupla Negação (DN)] $\phi\leftrightarrow\neg\neg\phi$
 \item[Transposição (Transp)]
$(\phi\rightarrow\psi)\leftrightarrow(\neg\psi\rightarrow\neg\phi)$
 \item[Impl. Material (IM)] $(\phi\rightarrow\psi)\leftrightarrow(\neg\phi\vee\psi)$
 \item[Tautologia (Taut)] $\phi\leftrightarrow(\phi\vee\phi)$
 \item[Tautologia (Taut)] $\phi\leftrightarrow(\phi\wedge\phi)$
\end{description}

\end{multicols}


\section{Exercícios}
\renewcommand{\theenumi}{\roman{enumi}}

\subsection{Tabelas Verdade}

\begin{exrcc}
Faça a tabela verdade dos seguintes conectivos:
\begin{enumerate}
\setlength{\itemsep}{0pt}
 \item Disjunção Exclusiva
 \item Nem $\phi$, nem $\psi$
\end{enumerate}
\end{exrcc}

\begin{exrcc}
Com a negação e qualquer conectivo binário é possível definir os demais
conectivos. Por exemplo, a implicação $\phi\rightarrow\psi$ pode ser definida em termos
da conjunção como $\neg(\phi\wedge\neg\psi)$, pois ela significa que o antecedente não
pode ocorrer sem o consequente. A tabela verdade das duas fórmulas é exatamente a mesma.
Sendo assim, reescreva as seguintes fórmulas usando somente os conectivos indicados
mais a negação. 

\begin{enumerate}
\setlength{\itemsep}{0pt}
 \item $\phi\rightarrow\psi$, disjunção
 \item $\neg(\phi\vee\neg\neg\psi)$, implicação
 \item $\phi\rightarrow\psi$, implicação
 \item $\phi\leftrightarrow\neg\psi$, disjunção
\end{enumerate}

\end{exrcc}






\begin{exrcc}
Faça a tabela verdade das seguintes fórmulas e determine se elas são tautologias,
contradições ou frases contingentes
\label{tabvcompl}


\begin{exmpl}
Exemplo com 3 proposições, para mostrar que a fórmula
$(\phi\vee(\psi\vee\sigma))\leftrightarrow((\phi\vee\psi)\vee\sigma)$ é uma tautologia:

\begin{center}
% use packages: array
\begin{tabular}[c]{lllllllllll}
$(\phi$ & $\vee$ & $(\psi$ & $\vee$ & $\sigma))$ & $\leftrightarrow$ & $((\phi$ & $\vee$
& $\psi)$ & $\vee$ & $\sigma)$ \\
V & V & V & V & V & V & V & V & V & V & V \\ 
V & V & V & V & F & V & V & V & V & V & F \\ 
V & V & F & V & V & V & V & V & F & V & V \\ 
V & V & F & F & F & V & V & V & F & V & F \\ 
F & V & V & V & V & V & F & V & V & V & V \\ 
F & V & V & V & F & V & F & V & V & V & F \\ 
F & V & F & V & V & V & F & F & F & V & V \\ 
F & F & F & F & F & V & F & F & F & F & F
\end{tabular}
\end{center}
\end{exmpl}


\begin{enumerate}
\setlength{\itemsep}{0pt}
 \item $(\neg{}p\vee{}q)\leftrightarrow(q\rightarrow{}p)$
 \item $(\phi\rightarrow\neg\phi)\leftrightarrow\neg\phi$
 \item $\neg(\phi\rightarrow\neg\phi)$
 \item
$(\phi\rightarrow(\psi\rightarrow\sigma))\leftrightarrow((\phi\wedge\psi)\rightarrow
\sigma)$
\item $(\phi\vee\psi)\rightarrow\neg(\phi\wedge\psi)$
\item $((\phi\vee\psi)\wedge\neg\psi)\rightarrow\phi$
\item $((\phi\vee\psi)\wedge\psi)\rightarrow\neg\phi$
\item $((\phi\vee\psi)\wedge\neg\psi)\rightarrow\neg\phi$
\item $\neg(\phi\vee\psi)\leftrightarrow(\neg\phi\wedge\neg\psi)$
\item
$((\phi\wedge\psi)\vee\sigma)\leftrightarrow(\sigma\rightarrow(\psi\rightarrow\phi))$
\item
$((\phi\wedge\psi)\rightarrow\sigma)\leftrightarrow(\psi\rightarrow(\phi\rightarrow
\sigma))$
\item
$((\phi\rightarrow\sigma)\wedge(\psi\rightarrow\sigma))\leftrightarrow((\phi\vee\psi)
\rightarrow\sigma)$
\end{enumerate}
\end{exrcc}

\vspace{3em}
Um método mais rápido para testar se uma determinada fórmula é uma tautologia ou não é
fazer a tabela-verdade de trás pra frente tentando determinar se há uma combinação de
valores que torna a fórmula falsa, quer dizer, ao invés de atribuir os valores das
partes e ir calculando o valor do todo, como na maneira comum de fazer as
tabelas-verdade, atribuímos o valor F para a fórmula toda e tentamos determinar que
valores as partes teriam. Se for possível determinar o valor de todas as partes, a
fórmula é falsificável, isto é, existe uma combinação de valores que a torna falsa e,
portanto, ela não é uma tautologia. Se, durante o processo, acontece de atribuirmos os
valores V e F para uma mesma fórmula, incorremos em contradição e se nos contradizemos
ao tentar falsificar uma fórmula isso significa que a fórmula é uma verdade lógica.

\begin{exmpl}
\label{testcont}

Testando uma fórmula contingente: $(\phi\vee\psi)\rightarrow(\psi\wedge\phi)$

\begin{description}
 \item[Primeiro Passo] Atribuímos F à fórmula toda. Como o operador principal é uma
implicação e só há um caso em que a implicação é falsa, sabemos que o antecedente é V e
o consequente, F.
\item[Segundo Passo] Procuramos determinar se há valores de $\phi$ e $\psi$ tais que
$\phi\vee\psi$ seja verdadeira e $\psi\wedge\phi$ seja falsa. A primeira fórmula é
verdadeira em três casos: $\langle{}V,V{}\rangle$, $\langle{}V,F{}\rangle$,
$\langle{}F,V{}\rangle$.\footnote{Esses símbolos são pares ordenados dos valores das
fórmulas. No caso, o primeiro item é o valor de $\phi$ e o segundo o valor de $\psi$.} A
segunda fórmula é falsa em três casos: $\langle{}V,F{}\rangle$, $\langle{}F,V{}\rangle$,
$\langle{}F,F{}\rangle$
\item[Terceiro Passo] Há duas combinações de valores -- $\langle{}V,F{}\rangle$,
$\langle{}F,V{}\rangle$ -- que tornam a primeira fórmula verdadeira e a segunda, falsa,
tornando o todo falso. Portanto a frase é falsificável e, portanto, \emph{não} é uma
tautologia
\end{description}

\end{exmpl}

\begin{exmpl}
\label{testtauto}
Testando uma fórmula tautológica: $(\phi\wedge\psi)\rightarrow(\psi\vee\phi)$

Como acima. Para a fórmula toda ser falsa $(\phi\wedge\psi)$
deve ser verdadeira e $(\psi\vee\phi)$, falsa. Há só um caso em que
$(\phi\wedge\psi)$ é verdadeira: quando $\phi$ e $\psi$ são verdadeiras. E também só há
um caso em que $(\psi\vee\phi)$
é falsa: quando ambas os disjuntos são falsos. Ora, nota-se aí claramente que para a
frase inteira ser falsa seria preciso que tanto $\phi$ quanto $\psi$ fossem verdadeiras
e falsas ao mesmo tempo, o que não é possível. Logo, não é possível a frase inteira ser
falsa, donde se segue que ela é uma tautologia. 

 
\end{exmpl}

\begin{exrcc}
Pratique esse método com as fórmulas do exercício \ref{tabvcompl}
\end{exrcc}

\begin{exrcc}
 Use esse método para testar se as fórmulas abaixo são uma tautologia ou não:

\begin{enumerate}
\renewcommand{\theenumi}{\roman{enumi}}
 \item $(\phi\wedge(\psi\rightarrow\phi))\rightarrow\psi$
 \item $(\phi\vee\psi)\rightarrow(\sigma\rightarrow\sigma)$
 \item $(\phi\wedge(\psi\vee\neg\phi)$
\end{enumerate}

 
\end{exrcc}

\begin{exrcc}
 Seguindo a lógica do método acima, qual seria o método para mostrar que uma
determinada fórmula é uma contradição?
\end{exrcc}


\subsection{Demonstração}

O símbolo $\vdash$ representa a relação de implicação entre premissas e conclusão ou,
em outras palavras, que a conclusão, no lado direito do símbolo, pode ser demonstrada a
partir das premissas, no lado esquerdo do símbolo, mediante a aplicação sucessiva das
regras de inferência expostas acima. 
Assim, por exemplo, a expressão $\{p, q\}\vdash{}p\wedge{}q$ expressa a idéia de que a
fórmula $p\wedge{}q$ é derivável do conjunto de premissas $\{p, q\}$ da mesma forma que
a expressão $\{p\rightarrow{}q, p\}\vdash{}q$ diz que as premissas $\{p\rightarrow{}q,
p\}$ permitem inferir a fórmula $q$, o que é feito por meio da aplicação da regra do
\emph{modus ponens}. Quando o símbolo é usado sem nenhum conjunto de premissas do lado
esquerdo, isso significa que a fórmula é derivável de um conjunto vazio de premissas,
ou seja, que ela é demonstrável mesmo sem assumir como dada nenhuma outra frase e,
portanto, é uma verdade lógica. Por exemplo, $\vdash{}p\rightarrow{}p\vee{}q$. É
interessante notar que para deduzir essa fórmula é preciso assumir $p$ como hipótese,
para então aplicar a regra da introdução da disjunção e, em seguida, a regra da
introdução da implicação e ``descartar'' a hipótese:

 \AxiomC{$[p]^{1}$}
 \RightLabel{\scriptsize $I\vee$}
 \UnaryInfC{$p\vee{}q$}
\RightLabel{\scriptsize $I\rightarrow$, descarte $1$ }
\UnaryInfC{$p\rightarrow{}p\vee{}q$}
 \DisplayProof

Quando as hipóteses são
descartadas (incorporadas a uma implicação), elas não constam no conjunto de premissas;
quando não são descartadas, têm
que constar. 

\begin{exmpl}
\label{bigdemo}

Provar que $\{(\phi\vee\neg\psi)\vee\sigma,
\neg\phi\vee(\psi\wedge\neg\phi)\}\vdash\psi\rightarrow\sigma$. Aplicando a regra da
distribuição à segunda premissa, obtemos
$(\neg\phi\vee\psi)\wedge(\neg\phi\vee\neg\phi)$. Se temos uma conjunção, podemos
inferir uma parte dela: $\neg\phi\vee\neg\phi$. Ora, isso é equivalente a $\neg\phi$
somente. Aplicando a regra da associação à primeira premissa obtemos
$\phi\vee(\neg\psi\vee\sigma)$. Aplicando a regra do silogismo disjuntivo a essa
conclusão e à conclusão anterior -- $\neg\phi$ -- obtemos $\neg\psi\vee\sigma$. E
usando a equivalência da Implicação Material, obtemos $\psi\rightarrow\sigma$, que é o
que queríamos demonstrar. 

\AxiomC{$\neg\phi\vee(\psi\wedge\neg\phi)$}
\RightLabel{\scriptsize $Dist$}
\UnaryInfC{$(\neg\phi\vee\psi)\wedge(\neg\phi\vee\neg\phi)$}
\RightLabel{\scriptsize $\wedge{}E$}
\UnaryInfC{$\neg\phi\vee\neg\phi$}
\RightLabel{\scriptsize $Taut$}
\UnaryInfC{$\neg\phi$}
\AxiomC{$(\phi\vee\neg\psi)\vee\sigma$}
\RightLabel{\scriptsize $Assoc$}
\UnaryInfC{$\phi\vee(\neg\psi\vee\sigma)$}
\RightLabel{\scriptsize $SD$}
\BinaryInfC{$\neg\psi\vee\sigma$}
\RightLabel{\scriptsize $IM$}
\UnaryInfC{$\psi\rightarrow\sigma$}
\DisplayProof
\end{exmpl}

\begin{exrcc}
 \label{demformal}


Usando as regras de inferência e as equivalências, demonstre as seguintes
implicações. Lembre-se: quando o conjunto de premissas for vazio, todas as hipóteses
devem ser descartadas, isto é, transformadas em antecedentes de uma implicação. Quando
o conjunto de premissas não for vazio, as mesmas podem ser usadas sem descarte

\begin{enumerate}
\setlength{\itemsep}{0pt} 
\item $\vdash(p\rightarrow{}q)\rightarrow(\neg{}q\rightarrow\neg{}p)$
 \item $\{p\rightarrow\neg{}q,{}q\vee{}r\}\vdash{}p\rightarrow{}r$
 \item $\{\neg\phi\vee\psi, \phi\}\vdash{}\psi$
 \item $\vdash\phi\rightarrow(\psi\rightarrow\phi)$
 \item $\vdash(\phi\wedge\psi)\rightarrow\neg(\phi\rightarrow\neg\psi)$
 \item $\neg(\phi\wedge\neg\psi), \phi\vdash\psi$
 \item $\{a\rightarrow{}b, b\rightarrow{}c, c\rightarrow{}a,
a\rightarrow{}\neg{}a\}\vdash\neg{}a\wedge\neg{}c$
\item $\{(\epsilon\vee\gamma)\rightarrow(\delta\wedge(\beta\wedge\lambda)),
\gamma\}\vdash\delta\wedge\beta$
\end{enumerate}
\end{exrcc}




\emph{Formalize os seguintes argumentos, isto é, construa uma prova formal deles}

\textbf{Exemplo}: Ou o gerente não notou  a mudança, ou ele a aprova. Ora, ele notou a
mudança muito bem. Portanto, ele deve aprová-la.

Primeira premissa: \textit{Ou o gerente não notou  a mudança, ou ele a aprova}.--
$\neg\phi\vee\psi$

Segunda premissa: \textit{Ele notou a mudança} -- $\phi$

Conclusão: \textit{Ele a aprova} -- $\psi$

Prova:\footnote{Se parecer estranho a alguém a aplicação do silogismo disjuntivo
note-se que $\neg\neg\phi$ é a negação de $\neg\phi$ e, portanto, no fundo a forma do
silogismo disjuntivo $\{\neg\phi\vee\psi,\neg\neg\phi\}\vdash\psi$ é a mesma do
silogismo $\{\phi\vee\psi,\neg\phi\}\vdash\psi$}

\AxiomC{$\phi$}
 \RightLabel{\scriptsize $\phi\leftrightarrow\neg\neg\phi$}
 \UnaryInfC{$\neg\neg\phi$}
 \AxiomC{$\neg\phi\vee\psi$}
\RightLabel{\scriptsize Sil.Disj.}
\BinaryInfC{$\psi$}
 \DisplayProof


\begin{enumerate}
 \item O oxigênio do tubo ou combinou-se com o filamento para formar óxido ou
evaporou-se completamente. O oxigênio do tubo não pode ter-se evaporado completamente.
Portanto, o oxigênio do tubo combinou-se com o filamento.
\item Se um estadista compreende que suas opiniões anteriores eram erradas e não
altera sua política, torna-se culpado de enganar seu povo; se altera a política,
expõe-se a que o acusem de vira-casaca. Obviamente, das duas uma: ou ela muda de
política, ou não muda. Portanto, resulta que ou ele é culpado de enganar seu povo ou
expõe-se à acusação de vira-casaca.
\item Se a cidadania romana tivesse sido uma garantia das liberdades civis, os
cidadãos romanos teriam gozado de liberdade religiosa. Se os cidadãos romanos tivessem
gozado de liberdade religiosa, então, os primeiros cristãos não teriam sido
perseguidos. Mas os primeiros cristão foram perseguidos. Portanto, a cidadania romana
não pode ter sido uma garantia de direitos civis.
\item Se a vítima tinha dinheiro nos bolsos, então o motivo do crime não foi roubo. Mas
o motivo tem que ter sido roubo ou vingança. Portanto, o motivo do crime deve ter sido
vingança.
\item Se Jones receber a mensagem, ele virá, desde que esteja interessado. Embora não
tenha vindo, ainda está interessado. Portanto, não recebeu a mensagem.

\end{enumerate}

\begin{exrcc}
 
Os seguintes argumentos estão incompletos, no sentido de que sua validade depende
de premissas implícitas. Formalize os argumentos e indique que premissas são
essas\footnote{Você pode forçar um pouco a sinonímia entre as frases a fim de tornar
os argumentos mais simples, mas também não exagere}:

\begin{enumerate}
 \item Se o caixa tivesse apertado o botão de alarme, o cofre forte teria se
fechado e a polícia teria sido avisada imediatamente. Se a polícia tivesse sido
avisada logo, teria pego os assaltantes. Por isso, o caixa não deve ter apertado o
alarme.
\item Obviamente só há duas alternativas: ou o criminoso é estranho à família e veio de
fora da residência ou ele já estava dentro da casa e, portanto, alguém de confiança
da família está envolvido. Mas a porta não foi arrombada, donde se segue que se o
assaltante veio de fora então contou com a ajuda de alguém de dentro da casa. Portanto,
de um jeito ou de outro, alguém de confiança da casa está envolvido.
\item Se eu pagar ao alfaiate, ficarei sem dinheiro, mas preciso de dinheiro para levar
minha noiva ao baile e se ela não for ao baile vai ficar muito chateada comigo. Mas,
 sem o terno não posso ir ao baile. Portanto, de um jeito ou de outro
minha noiva irá ficar chateada comigo!
\item Deus não existe, porque a existência do Mal é incompatível com a existência de
um ser onisciente, onipotente e bondoso. 
\end{enumerate}
\end{exrcc}

\subsection{Validade Semântica}

Nós vimos acima a aplicação de uma noção de demonstração que não recorre às noções de
verdade ou falsidade: uma fórmula é derivável de outra(s) segundo essa concepção se ela
pode ser obtida dessa(s) outra(s) a partir da aplicação das regras de inferência. Mas
também temos uma noção semântica de consequência lógica, a que aprendemos no começo do
curso: uma fórmula se segue de um conjunto de premissas se não pode ser falsa quando
estas são verdadeiras. Assim, temos um método simples para testar a validade de um
argumento é testar se é possível ocorrer de as premissas serem verdadeiras e a
conclusão, falsa. Em termos da lógica proposicional isso significa: o argumento será
\emph{inválido} se existir uma atribuição de valores-verdade às frases atômicas que
torne as premissas verdadeiras e a conclusão, falsa. E, inversamente, será
\emph{válido} se não for possível atribuir um valor-verdade para cada frase atômica sem
entrar em contradição. Quer dizer, quando pegamos um argumento válido e tentamos
encontrar valores de verdade que tornam as premissas verdadeiras e a conclusão falsa
acabamos tendo que atribuir valores-verdade contrários a uma mesma frase. A técnica é
parecida com a utilizada para testar se uma frase é uma tautologia ou não (ver exemplos
\ref{testcont} e \ref{testtauto} ), só que agora fazemos com várias frases e seus
componentes.

\begin{exmpl}
Mostrar que o argumento $\{A\rightarrow{}B, C\rightarrow{}D,
A\vee{}D\}\vdash{}B\vee{}C$ é inválido.

Se o argumento é inválido é possível encontrar valores de A, B, C e D tais que as três
premissas -- $A\rightarrow{}B, C\rightarrow{}D,
A\vee{}D$ -- sejam verdadeiras e a conclusão -- $B\vee{}C$ -- falsa. Raciocinemos. Se a
conclusão $B\vee{}C$ for falsa, tanto $B$ quanto $C$ são falsas, pois esse é o único
caso em que a disjunção é falsa. Se $B$ é falsa, $A\rightarrow{}B$ só pode ser
verdadeira se $A$ for falsa. E se $C$ é falsa, então $C\rightarrow{}D$ será
verdadeira, não importando o valor-verdade de $D$. Por fim, se $A$ é falsa, então, para que
$A\vee{}D$ seja verdadeira, é preciso que $D$ seja verdadeira. Com isso definimos valores para 
cada frase atômica tais que as premissas se tornam verdadeiras e a conclusão, falsa, mostrando, 
assim, que o argumento é inválido. Colocando o exemplo numa tabela:

\begin{center}
% use packages: array
\begin{tabular}[c]{cccccccc}
$A$ & $B$ & $C$ & $D$ & $A\rightarrow{}B$ & $C\rightarrow{}D$ & $A\vee{}D$ & $B\vee{}C$
\\
\hline
F	&	F	&	F	& V &	V	&	V	&
V	&	F

\end{tabular}
\end{center}


\end{exmpl}

\begin{exmpl}
Mostrar que o argumento $\{(\alpha\vee\beta)\rightarrow(\gamma\wedge\delta), 
\neg\gamma\}\vdash\neg\alpha$ é válido.

Se o argumento é válido, não conseguiremos evitar a contradição ao tentarmos determinar
valores de $\alpha$, $\beta$, $\gamma$ e $\delta$ tais que as premissas sejam
verdadeiras e a conclusão, falsa. Raciocinemos. Se a conclusão $\neg\alpha$
é falsa, então $\alpha$ é verdadeira. Se $\alpha$ é verdadeira, então $\alpha\vee\beta$
é verdadeira e se é assim, a premissa
$(\alpha\vee\beta)\rightarrow(\gamma\wedge\delta)$ só pode ser toda verdadeira se
$\gamma\wedge\delta$ também for, porque esse é o único caso em que uma implicação com
antecedente verdadeiro é verdadeira. Se $\gamma\wedge\delta$ é verdadeira, então cada
parte da conjunção é verdadeira. Assim $\gamma$
é verdadeira. Porém, se a segunda premissa -- $\neg\gamma$ -- também é verdadeira,
$\gamma$ teria que ser falsa! Assim, para que as duas premissas fossem verdadeiras e a
conclusão, falsa, seria preciso que $\gamma$ fosse ao mesm tempo verdadeira e falsa, o
que não pode ocorrer. Portanto, não é possível afirmar as premissas (atribuir valor V a
elas) e negar a conclusão (atribuir valor F a ela) sem cair em contradição. Logo, o
argumento é válido. Fazendo a tabela:

\begin{center}
\begin{tabular}[c]{cccccccccc}
$\alpha$ & $\beta$ & $\gamma$ & $\delta$ & $\alpha\vee\beta$ & $\gamma\wedge\delta$ &
$(\alpha\vee\beta)\rightarrow(\gamma\wedge\delta)$ & $\neg\gamma$ & $\neg\alpha$ \\
\hline
V	& V/F	&	{\upshape\textbf{X}}	&	V	&	V	&
V & V & V	&	F
 
\end{tabular}
\end{center}

Marque com um X a contradição que você encontrar. Basta encontrar uma contradição para
que se mostre a impossibilidade de afirmar as premissas e negar a conclusão. 
 
\end{exmpl}

\begin{exrcc}
 Pratique o método acima provando a validade dos argumentos do exercício
\ref{demformal}. 
\end{exrcc}


\begin{exrcc}

Usando o método acima, demonstre a validade ou invalidade dos argumentos abaixo:

\begin{enumerate}
\setlength{\itemsep}{0pt}
 \item $\{\phi\vee\neg\psi, \neg(\neg\sigma\wedge\tau),
\neg(\neg\phi\wedge\neg\tau)\}\vdash\neg\psi\rightarrow\sigma$
\item $\{M\rightarrow(N\vee{}O), N\rightarrow(P\vee{}Q), Q\rightarrow{}R,
\neg(R\vee{}P)\}\vdash\neg{}M$
\item
$\{y\rightarrow{}z,\,{}z\rightarrow(y\rightarrow(r\vee{}s)),\,r\leftrightarrow{}s,\,
\neg(r\wedge{}s)\}\vdash\neg{}y$
\item $\{(D\wedge{}E)\rightarrow{}F,\,
(D\rightarrow{}F)\rightarrow{}G\}\vdash{}E\rightarrow{}G$
\item $\{\neg(e\wedge{}f),\, (\neg{}e\wedge\neg{}f)\rightarrow(g\wedge{}h),\,
h\rightarrow{}g\}\vdash{}g$
\end{enumerate}

 
\end{exrcc}

\pagebreak

\begin{exrcc}
O objetivo do curso de lógica é chegar gradualmente na análise de argumentos reais, que
são muito mais enganosos do que os exemplos mais domesticados dos exercícios acima. Os
casos abaixo envolvem várias cascas de banana. Responda às perguntas e justifique suas
respostas usando as técnicas da lógica. No caso de argumentos inválidos, mostre um
contraexemplo, um argumento da mesma forma, que siga o mesmo raciocínio, mas que tenha
premissas verdadeiras e conclusão, falsa. 

\begin{enumerate}
\setlength{\itemsep}{0pt}
 \item Se a pena de morte detivesse assassinatos, ela seria justificada. Mas dado que
ela não detém tais crimes, então se segue que ela não é justificada?
\item Suponha que é verdadeiro que se Zezinho estudar filosofia hoje à noite, ele vai
reprovar o teste de matemática amanhã e se ele, ao contrário, estudar matemática, ele
irá reprovar em seu teste de filosofia. Suponha também que ele não pode estudar para os
dois testes. Disso se segue que Zezinho irá reprovar em pelo menos um dos testes amanhã?
\item Minha colher está seca e minha colher estaria molhada se eu tivesse mexido meu
café. E eu não teria mexido meu café a menos que eu tivesse posto açúcar nele. Então,
eu não devo ter colocado açúcar no café, certo?
\item Maria diz que não vai pra cama com João a menos que ele se casem. João concorda
em se casar, mas, na lua de mel, Maria ainda se recusa a ir para a cama com ele. Será
que ela quebrou sua promessa?
\item Como sabemos, esferas projetam sombras curvas e a Terra projeta uma sombra curva
sobre a Lua durante os eclipses lunares. Isto prova que a Terra é esférica?
\item O presidente da IBM certamente é um homem influente. No entanto, ele não
conseguiu matricular sua filha na Universidade Whatsamatta. Portanto, é falso, como
muita gente vem dizendo, que só pessoas com influência conseguem uma vaga na
Whatsamatta para os seus filhos. Certo?
\item Uma amiga sua diz ``As notícias de hoje em dia só me deixam deprimida e é ruim
ficar assim, portanto é ruim ficar assistindo notícias''. O argumento dela é válido?
\item Você ouve uma pessoa dizendo o seguinte: ``Fulana é uma hipócrita: ela sempre diz
que o aprendizado é mais importante que a nota, mas ontem passou o dia todo fazendo um
exercício para nota extra que ela já sabia fazer e que não ia acrescentar em nada ao
que ela já sabia!''. A pessoa tem razão no seu juízo sobre Fulana?
\item Alceu diz: ``Os únicos que falam a verdade aqui somos eu e Catulo''. Safo diz
``Catulo é um mentiroso''. Catulo replica: ``Safo diz a verdade. Ou então é Alceu quem
mente''. Assumindo que quem mente, mente sempre e quem diz a verdade, diz a verdade
sempre, que está mentindo e quem está falando a verdade?
\item Anaximandro diz ``Não se deve confiar em Heráclito''. Parmênides diz:
``Anaximandro e Heráclito nunca mentem''. Heráclito diz: ``Parmênides disse a
verdade''. Quem mente e quem diz a verdade?
\item No Senado Romano o seguinte debate acontece. Marco Antônio diz ``Ou foi Cassius ou
foi Brutus (ou ambos)''. Cassius responde: ``Não fui eu. Marco Antônio está mentindo''.
Brutus diz: ``Também não fui eu''. Só um dos três está falando a verdade. Então quem
foi o culpado? E quem está falando a verdade?
\end{enumerate}

\end{exrcc}




\end{document}
